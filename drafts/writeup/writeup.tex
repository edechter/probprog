\documentclass{article}

\usepackage{amsmath}

\title{Extensible Probabilistic Programming}
\author{Eyal Dechter \& Matt Johnson}

\begin{document}

\section{Overview of Probabilistic Programming}
    The goal of probabilistic programming is to enable the user to perform probabilistic inference by writing down a probability distribution in a purely declarative fashion. Thus, probabailistic program is to probabilistic inference what logic programming is to first order logic: instead of specifying what the computer should do, the user specifies constraints on the computer's output, leaving hidden the mechanism and algorithm by which the computer finds the answer. In logic
    programming, the user writes down a set of logical statements and can then ask various queries of the system. Similarly, in probabilistic programming, the goal is to be able to accomodate various queries from the declared probability distribution, such as samples from the distributions and statistics related to the distribution.  
    :w
\section{The Metropolis-Hastings Algorithm}

\section{The Metropolis-Hastings Algorithm for program Traces}

\section{Conjugate pairs}

\end{document}
